% !TeX root = ../main.tex

\begin{acknowledgements}
  两年的硕士生涯转瞬即逝。回想大四那年,经过慎重考虑,我选择了来外校读研,虽然知道前进道路充满着未知和困难,依旧满怀热情踏上了读研的道路。读研期间有过彷徨,经历过太多迷茫与困难,但是最终也得到了历练。如今毕业在即,回首过去的两年,发现许多人都给予了我很大的帮助。特在此进行由衷的感谢。
  
  首先要感谢我的导师朝红阳教授,本论文的是在她的耐心指导下完成的。朝老师治学严谨,对待科研从来都是一丝不苟。朝老师经常强调做科研要讲究逻辑的严谨性,要把问题定义清楚,认准了方向就要坚持做下去,不要轻言放弃。疫情期间,朝老师坚持线上每周给大家开组会,为大家的科研指明方向。老师始终强调,硕士无论专业硕士还是学术硕士,都要有完整的投稿过程,这极大地促进了实验室同学们科研的积极性。老师的目的是想让他的学生经历科研写作这样一套科学地训练体系,这样以后无论是继续科研还是工作都会有清晰的思路去研究新领域的问题。经过两年的硕士生涯,越发对老师的观点表示强烈认同。
  
  感谢丁圣勇师兄对我的帮助,师兄带我入门 SLAM 领域,通过对 SLAM 的学习,极大地提升了我的工程能力。丁师兄为我的投稿过程操了很多心,每天与我同步工作,指导我修改论文,帮助我完善新视图合成领域的知识体系。同样要感谢图语公司的柯志麟同学和姚王泮同学,在我大四和研一上学期,帮助我入门计算机视觉领域。
  
  感谢实验室的谷溢、仁杰、鸿鑫、泽林同学,大家一起科研的时光很快乐,研究生生涯得到他们的很多帮助,从他们身上学习到很多知识。与谷溢同学合作的 E-NeRF 论文虽然没中,但确实是我们俩以及丁圣勇师兄的心血,很怀念那段时光。谷溢同学对我的科研生涯帮助颇大,耐心帮助我分析实验,调试代码。鸿鑫师兄在我毕业论文写作过程给予了很多帮助,包含如何定义问题,如何写好工程论文,如何保证论文的逻辑严谨性。感谢汇国师兄,牺牲自己的假期时间帮我修改论文的摘要和绪论,让我对文章逻辑的完整性有了深刻的体会。
  
  最后感谢我的女朋友,始终支持我鼓励我,这才让我有了坚持读研的动力。还要感谢我的父母一直以来对我的默默付出。
\end{acknowledgements}
