% !TeX root = ../main.tex

\begin{resume}

   \subsection*{学术论文}

   \begin{achievements}
   	\item Gu Y, Zhou X H, Li R J, et al. Enhancing Neural Radiance Fields With Joint Pose Optimization[C]. IEEE International Conference on Multimedia and Expo (ICME), 2021. (录用边缘,第二作者)
%   	\item International Conference on Multimedia and Expo (ICME), 2021. (录用边缘,第二作者)
   	
%     \item Yang Y, Ren T L, Zhang L T, et al. Miniature microphone with silicon-based ferroelectric thin films[J]. Integrated Ferroelectrics, 2003, 52:229-235.
%     \item 杨轶, 张宁欣, 任天令, 等. 硅基铁电微声学器件中薄膜残余应力的研究[J]. 中国机械工程, 2005, 16(14):1289-1291.
%     \item 杨轶, 张宁欣, 任天令, 等. 集成铁电器件中的关键工艺研究[J]. 仪器仪表学报, 2003, 24(S4):192-193.
%     \item Yang Y, Ren T L, Zhu Y P, et al. PMUTs for handwriting recognition. In press[J]. (已被Integrated Ferroelectrics录用)
   \end{achievements}


  \subsection*{专利}

  \begin{achievements}
%  	\item 中国公开发明专利.2021.(第一发明人,与本文第三章相关)
    \item 周晓豪, xxx. 一种基于符号距离函数的三维重建方法[P]: 中国公开发明专利, 202011227999.5 (第一发明人,与本文第三章相关)
    
    
    % \item 任天令, 杨轶, 朱一平, 等. 硅基铁电微声学传感器畴极化区域控制和电极连接的方法: 中国, CN1602118A[P]. 2005-03-30.
    % \item Ren T L, Yang Y, Zhu Y P, et al. Piezoelectric micro acoustic sensor based on ferroelectric materials: USA, No.11/215, 102[P]. (美国发明专利申请号)
  \end{achievements}

\end{resume}
