% !TeX root = ../main.tex

\chapter{结构排版及字体规范}
\section{论文印制规格}
学位论文一律采用 A4纸张双面打印,纸的四周留足空白边缘,以便装订、复制和读者批注。

\section{中英文摘要}
硕士学位论文摘要一般不超过1200字, 博士学位论文一般不超过2000字。
关键词三到五个,用逗号分隔。

\section{目录页}
\begin{enumerate}
    \item 目录应两端对齐;
    \item 目录页排版只排到二级标题, 即章和节。
\end{enumerate}

\subsection{测试目录}
三级标题 \verb|\subsection{}| 不应出现在目录当中。

\section{主体部分}
\begin{enumerate}
    \item 章的标题应局中, 采用小二号黑体; 节的标题左边空两格, 小三号宋体, 加粗。 文章段落内容采用小四号宋体。
    \item 章与节的题目之间空两行。
    \item 节标题与段落内容之间空一行。
    \item 关于关使用文字、数字的书写法:
        \begin{enumerate}
            \item 应用汉语简化字书写。
            \item 世纪、年份一概用阿拉伯数字书写,并写全数。例: 20 世纪 90 年代;1998 年不能写成 98 年。
            \item 公式均需标注公式号,公式号用圆括号,阿拉伯数字表示,按章编排。 \\
                例:第二章第1公式编为:
                \begin{equation}
                    \begin{aligned}
                        X + Y = Z.
                    \end{aligned}
                \end{equation}
            \item 论文中的物理量、量纲及符号均采用国际标准 (SI) 和国家标准 (GB)。
        \end{enumerate}
        关于公式, 符号的说明详见第\ref{equations}章。
\end{enumerate}
